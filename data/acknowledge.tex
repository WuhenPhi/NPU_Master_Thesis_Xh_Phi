
\Thanks

时光飞逝,转眼间已经要说再见,在西工大度过的七年时光历历在目,有着太多的不舍。学习和成长的过程,离不开老师的悉心教导和同学的关心帮助,在此我要向大家表示我最真诚的感谢。

首先要特别感谢我的两位指导老师:张雯教授和陈景东教授。从论文选题到研究工作的开展以及论文的撰写,都是在两位老师的指导下进行的。两位老师认真负责、精益求精的工作态度、渊博的知识储备、平易近人的处事方法深深影响了我,永远是我科研和生活的人生榜样。从本科毕业设计到现在,这三年多的时间里,我从刚开始的科研小白到现在有一定的理解和成果,都离不开两位老师的耐心指导,一次次的讨论让我对自己的研究方向和工作内容有了清晰的认知,使我的学习能力与动手能力得到了提升。老师们不仅在科研上给予帮助,还非常关心我们的生活情况和就业进展。

感谢教研室的师兄师姐和师弟师妹们,他们在整个研究生生涯中给予了我很多帮助,一起度过了很多快乐的时光。在探讨交流过程中,我吸收了很多宝贵的经验,为科研工作奠定基础。其中特别感谢郗经纬同学,在遇到问题时共同讨论、互帮互助,还有同级的陈奕彤和李梦真,在整个论文的撰写过程中相互鼓励,正是这种合作精神,给我们的研究生生涯画上了一个完美的句号。感谢张俊卿师兄在我遇到不懂的问题时给予分析和建议,对论文提出修改意见,同时感谢帮我审阅论文的师妹尤新雅、胡轩琦、师弟王佳乐,感谢你们提出的宝贵意见。感谢我硕士期间的宝藏舍友们,每天回到宿舍都会有一个开心轻松的氛围,互相关心,相信我们七年的友谊在之后的时间里可以得到更好的发展。

在此,还要感谢我的家人和男朋友对我无微不至的关心与理解。你们是我的坚强后盾,对我的任何决定都给予支持,让我可以毫无顾虑地勇往直前。在我遇到困难和压力时,你们的陪伴和鼓励使我能够保持乐观的心态,做出更好的成绩。

同时,我还要感谢我的班主任李道江老师和就业指导中心这个大家庭,在我的生涯规划和求职过程中给予指导和关心。最后还要向本领域进行相关研究的学者专家,以及各位审阅评议论文的老师表示由衷的感谢!