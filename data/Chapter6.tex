\chapter{ 总结与展望  }

\section{全文总结}

近年来,增强现实中的双耳渲染技术一直受到科学界众多学者的关注,在虚拟声学、建筑声学、语音通信、信息系统、媒体社交、教育和游戏娱乐等领域有着广泛的应用。
% 其主要包括对声场的录制及后续的双耳渲染算法。在声场录制方面,球形麦克风阵列凭借其三维结构及巧妙的麦克风配置,可以更好地采集声场的特征,具有高质量、高保真的录制效果,是多通道录制与重构、空间音频技术的重要工具。
本论文主要研究了基于球谐分解的双耳渲染技术,从录制声场和头相关传递函数两方面出发,对原始算法加以改进,进一步提升双耳信号的感知性能,并且使用球形麦克风阵列对所提出的算法进行评估。

本文的研究工作主要包括以下几个方面:

1. 对典型的双耳渲染算法进行了概述和总结,对本文所关注的基于球谐分解的双耳渲染算法及其存在的问题进行深入研究,并对声场的正交球谐分解等基础理论加以详细介绍,为后续工作奠定理论基础。

2. 研究了基于球谐分解的双耳渲染算法中声场和头相关传递函数的球谐分解及系数获取方法。在声场方面,分析了空心球和刚性球的特点,给出了正确获取声场系数所需的麦克风数目,以及两种阵列所获取的声场系数的区别,并说明了刚性球的优越性。在头相关传递函数方面,首先对其获取方式进行概述,并对本文所采用的两种数据库加以详细介绍,最后给出了球谐系数的获取方法。

3. 研究了头相关传递函数的预处理方法,并通过实验验证了该方法的有效性。本文采用与频率相关的相位对准预处理方法,基于高频处头相关传递函数的幅度部分对定位的感知起决定作用这一事实,对高频处的相位加以修正。该预处理方法一方面准确地表示头相关传递函数的幅度谱,并有效降低了头相关传递函数的分解阶次。最后在现有的头相关传递函数数据库上进行仿真实验,结果表明该预处理算法可以将头相关传递函数的分解阶次由原来的~34~阶降低至~15~阶。

4. 提出了一种新的声场扩阶理论,并且将声场扩阶理论、头相关传递函数预处理和基于球谐分解的双耳渲染算法相结合,实现了基于声场扩阶的双耳渲染算法,给出总体算法框架。该声场扩阶理论对麦克风阵列的采集声场进行分析,将入射声场分解为直达波和混响场的叠加。根据直达波入射方向进行空间加窗处理,可以实现声场分量的阶次提升,同时扩大控制区域的半径。除此之外,引入球谐域~MUSIC~定位算法以获取声源的入射方向,对采用的空域窗函数进行介绍并给出了相应的选取原则和球谐系数的获取方法。

5. 通过实验将本文所提出的算法与对标算法进行对比和分析,验证了本算法的有效性。在消声室和混响环境下,从双耳时间差、双耳声级差和双耳听觉互相关系数这三个评价指标出发,对双耳信号的声源定位精度和空间感知进行评估。实验结果表明,在不同的混响情况下、声源位于不同方向、采用不同声源类型时,本文所提出的双耳渲染算法均能达到较好的效果,优于现有算法。并且在消声室环境下,对空心球和刚性球在不同双耳渲染算法下的性能进行对比,分析空心球存在的零点问题及其危害,验证了刚性球的优越性。


\section{工作展望}

本文以基于球谐分解的双耳渲染算法为基础,在声场和头相关传递函数两方面对原始算法加以改进,提出了一种新的声场扩阶理论,并提出一种新的双耳渲染算法,进一步提升了双耳信号的感知性能。本文虽然取得了一定的成果,但是由于时间有限,仍有部分内容没有深入研究,主要集中在以下两个方面:

1. 目前本文所提出的双耳渲染算法还未在实际系统上进行,可以借助已有的实验设备,建立一个演示系统,并在此基础上进行更广范围的听音评测。

2. 未来随着个性化听觉感知的发展,双耳渲染技术与个性化头相关传递函数的结合很有必要,可以将本文采用的头相关传递函数数据库更换为听者特有的头相关传递函数。
