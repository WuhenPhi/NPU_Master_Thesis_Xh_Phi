\chapter{论文模板介绍}

\section{简介}
本论文模板用于撰写西北工业大学研究生学位论文,基于西北工业大学材料学院提供的模板编写,项目链接 \href{https://github.com/WuhenPhi/NPU_Master_Thesis_Xh_Phi}{NPU\_Master\_Thesis\_Xh\_Phi
} ,理论上适用于材料学院硕士和博士,其他学院可能会有不同要求,可以按需求更改。

\section{模板结构}
本论文模板采用模块化结构,主文档为 /NPUthesis.tex,所有论文结构均包含在该文件当中,子文档按照所需顺序采用命令包含进入,列表如下:
\begin{itemize}[leftmargin=*]
	\item 模板文件:/nputhesis.cls,其中包括文档字体字号,页眉页脚,章节编号,封面格式定义,图片/表格编号,参考文献,声明页定义等,有能力可以自行更改。
	\item 配置文件: /npuphd.cfg,包含论文模板中的固定文字部分,在cls文件中 导入使用,例如“西北工业大学硕士学位论文”、“题目”、“学科专业”等字样,如要更改为“博士”,或其他情况,可在此自行更改。
	\item /data 文件夹中包含论文中的内容部分列表如下:
	\begin{itemize}[leftmargin=*]
		\item /data/cover.tex 封面信息填写及 \textbackslash makecover 命令用于生成三个封面,通常[]中为英文信息,\{\}中为中文信息,\textbackslash SVB... 命令为控制封面中的副导师信息存在与否。
		\item /data/reviewerlist.tex 为论文评阅人名单,填写时只需在对应表格位置更改即可。
		\item /data/abstract.tex 为中英文摘要,填写时在abstract 环境中填写中文摘要,在keywords 环境中填写中文关键词,时在Abstract 环境中填写英文摘要,在Keywords 环境中填写英文关键词,基金信息在\textbackslash support中填写
		\item /data/innovation.tex 为创新点,直接填写即可,编号手动,或自己调整\textbackslash itemize 格式。
		\item /data/chapterX.tex 根据数字不同分别为各章内容,每章中存在 \textbackslash chapter、\textbackslash section和\par \textbackslash subsection 三级,分别对应三级标题,各章内部无区别,可以随意调整顺序。
		\item /data/conclusion.tex 中为结论部分格式与章相同,同创新点类似,编号手动。
		\item /data/refs.bib 为论文中所有参考文献bibtex的合集。
		\item /data/achievement.tex 为成果页,论文和专利分别为两个列表,直接填写即可,如有参加会议等项目自行添加。
		\item /data/acknowledge.tex 为致谢页,直接填写内容即可。
	\end{itemize}
	\item /figure 文件夹下为各个章节的图片,建议图片分章放置以便管理,将每章图片放入对应文件夹即可,不建立子文件夹亦可,但图片数量变多时会比较杂乱。
	\item \textbackslash bibliographystyle{gbt7714-numerical} 和\textbackslash bibliography{data/refs} 分别用于设置参考文献格式为/gbt7714-numberical.bst和参考文件源文件为/data/refs.bib,参考文献格式来源于\par\href{https://github.com/zepinglee/gbt7714-bibtex-style}{gbt7714bibtex-style}项目,在其上做了部分更改如作者姓名缩写部分等。
	\item \textbackslash statement直接生成声明页,具体内容在npuphd.cfg中更改。
	\item \textbackslash frontmatter、\textbackslash mainmatter和\textbackslash backmatter 用于分割文档,会影响页码和章节编号等问题,不建议更改。
\end{itemize}

\section{免责声明}
{
\bfseries
学位论文中的具体格式请参考学校的最新规定,不一致的以学校为准

作者不承担因使用本模板产生的一切后果,使用即代表同意以上内容}
