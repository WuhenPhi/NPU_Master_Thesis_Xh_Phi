% !TeX program = xelatex 
%指定编译工具 xelatex
\documentclass[twoside,cjk,UTF8]{nputhesis}
%\documentclass[twoside,cjk,UTF8,draft]{nputhesis}%如果编译速度过慢,建议打开draft,可以明显加速因图片过大导致的速度慢,最后版本改回去就可以,打开draft后可能出现目录等处溢出块显示为黑色的情况
%\special{dvipdfmx:config z 0} %取消PDF压缩,加快速度,最终版本生成的时候最好把这句话注释掉,可能会让pdf体积增加10倍以上
\raggedbottom
\begin{document}
\schoolno{10699}
%\classno{}
%\secretlevel{}
\title[ Binaural Rendering Technology In Augmented Reality ]{	增强现实中\\双耳渲染技术研究}
\author[Jieru Chen]{陈洁茹}
\authorno{2018200480}
\major[Signal and Information Processing]{信号与信息处理}
\supervisor[Wen~Zhang~,~Jingdong~Chen]{张雯~~陈景东}
\applydate[March 2021]{2021~年~3~月}
\support{本文研究得到XX基金(编号:XXXXXXX)资助。}
\makecover%中文外封,中英文内封
\begin{titlepage}
\begin{center}
\pagestyle{empty}
\vspace*{40pt}
\heiti\zihao{3}学位论文评阅人和答辩委员会名单

\vspace*{30pt}

\heiti\zihao{4}学位论文评阅人名单
\begin{table}[H]
	\centering
	\zihao{-4}
	\renewcommand\arraystretch{1.3} 
	\begin{tabular*}{\hsize}{p{3.71cm}<{\centering}@{\extracolsep{\fill}}p{2.83cm}<{\centering}p{8.73cm}<{\centering}}
		\textbf{姓名}&	\textbf{职称}	&\textbf{工作单位}\\
		xxx&教授	&西北工业大学(明评示例)\\
			&	&	\\
		全盲评阅&无&无(盲评示例)\\
			&	&	\\
			&	&	\\
	\end{tabular*}
\end{table}
\vspace*{20pt}

\heiti\zihao{4}{答辩委员会名单}
\begin{table}[H]
	\centering
	\renewcommand\arraystretch{1.5} 
	\zihao{-4}
	\begin{tabular*}{\hsize}{p{3.76cm}<{\centering}p{2.51cm}<{\centering}p{2.25cm}<{\centering}p{6.75cm}<{\centering}}
		\textbf{答辩日期}&\multicolumn{3}{c}{2023年03月01日}\\
		\textbf{答辩委员会}&	\textbf{姓名}	&\textbf{职称}	&\textbf{工作单位}\\
		\textbf{主席}&	&	&	\\
		\textbf{委员}&	&	&	\\
		\textbf{委员}&	&	&	\\
		\textbf{委员}&	&	&	\\
		\textbf{委员}&	&	&	\\
		\textbf{秘书}&	&	&	\\
		
	\end{tabular*}
\end{table}
\end{center}
\end{titlepage}%评阅人名单
%
\frontmatter%这里会涉及标题,页码等问题,不要取消
% 中文摘要
\begin{abstract}
	
	声学场景的双耳渲染旨在给听众提供身临其境的体验,是虚拟现实、增强现实、沉浸式多媒体产品、虚拟声学等领域的重要研究课题。最近,基于球谐分解的双耳渲染算法得到了广泛关注。该算法对声场和头相关传递函数进行球谐函数展开,并且直接在球谐域进行渲染,避免了虚拟扬声器数目和位置对听感的影响。但是此类算法存在声场和头相关传递函数不匹配的问题,例如通常录音设备的尺寸小于人头尺寸,声场分解阶次(通常小于~5~阶)远小于头相关传递函数的分解阶次(34~阶)。这些不匹配问题导致定位线索的损伤、空间感的降低和音色的改变。
	
	本论文主要针对以上问题,从录制声场和头相关传递函数两方面出发,对原始算法加以改进。主要的研究内容包括以下四个方面:
	
	1. 深入研究了基于球谐分解的双耳渲染算法及其存在的问题。对声场和头相关传递函数的球谐分解及系数获取方法进行研究,针对空心球和刚性球展开讨论,给出了正确获取声场系数所需的麦克风数目,以及两种球阵所获取的声场系数的区别。
	
	2. 研究了头相关传递函数的预处理方法,并通过实验验证了该方法的有效性。该方法采用与频率相关的相位对准,保留低频相位信息的同时对高频的相位加以修正。和直接使用最小二乘方法求解头相关传递函数的球谐分量相比,该预处理方法一方面可以准确地表示头相关传递函数的幅度谱,另一方面有效降低了头相关传递函数的分解阶次。
	
	3. 提出了一种声场扩阶理论和一种新的双耳渲染算法。该理论对麦克风阵列的采集声场进行分析,将入射声场分解为直达波和混响场的叠加。根据直达波入射方向进行空间加窗处理,可以实现声场分量的阶次提升,同时扩大控制区域的半径。并且本研究将声场扩阶理论、头相关传递函数预处理和基于球谐分解的双耳渲染算法相结合,实现了基于声场扩阶的双耳渲染算法。
	
	4. 在消声室和混响环境下,从双耳时间差、双耳声级差和双耳听觉互相关系数这三个评价指标出发,将本文所提出的算法与对标算法进行对比和分析,验证本算法的有效性。实验结果表明,在不同的混响情况下、声源位于不同方向时,本文所提出的双耳渲染算法均能达到较好的效果,优于现有算法。并且对空心球和刚性球在不同双耳渲染算法下的性能进行对比,验证了刚性球的优越性。
	
	%\lipsum[2-3]
	\begin{keywords}
		双耳渲染、头相关传递函数、空间音频、球谐分解、声场扩阶
	\end{keywords}
\end{abstract}

% 英文摘要
\begin{Abstract}
	
	Binaural rendering of acoustic scenes aims to provide the audience with an immersive experience. It is an important research topic in the fields of virtual reality, augmented reality, immersive multimedia products, and virtual acoustics. Recently, binaural rendering algorithms based on spherical harmonic decomposition have received much attention. This algorithm performs spherical harmonic expansion of sound field and head-related transfer function(~HRTF~), and renders directly in spherical harmonic domain, avoiding the influence of the number and position of virtual speakers on hearing sense.
	However, there are some mismatches between the sound field and the head-related transfer function in this kind of algorithm. For example, the size of the recording equipment is usually smaller than the head size, and the decomposition order of the sound field (usually less than ~5~ order) is much smaller than the decomposition order of the head-related transfer function (~34~ order). These mismatches lead to impairments of localization cues, loss of spaciousness, and changes in timbre.
	
	Aiming at the above problems, this paper improves the original algorithm from two aspects, sound field and head-related transfer function. The main research contents include the following four aspects:
	
	1. The binaural rendering algorithm based on spherical harmonic decomposition and its problems are studied in detail. The spherical harmonic decomposition and coefficient calculation method of sound field and HRTF are studied, and the number of microphones which is needed to acquire sound field coefficient correctly. The difference of sound field coefficient obtained by open and rigid spherical array are discussed.
	
	2. The preprocessing method of head-related transfer function is studied and the effectiveness of this method is verified by experiments. In this method, the frequency-dependent phase alignment is adopted to retain the phase information of low frequency and modify the phase information of high frequency. Compared with the least square method, this method can not only accurately represent the magnitude spectrum of the head-correlation transfer function, but also effectively reduce the decomposition order of the head-correlation transfer function.
	
	3. A new theory for enlarging the order of sound field and a new binaural rendering algorithm are proposed. In this theory, the incident sound field is decomposed into the superposition of direct component and diffuse field.
	According to the direction of direct component, spatial masking can improve the order of the sound field and enlarge the radius of the control area. In addition,  a new binaural rendering algorithm is realised by combing combines this theory , the frequency-dependent phase alignment preprocessing method of the head-correlation transfer function and the binaural rendering algorithm based on spherical harmonic decomposition.
	
	
	4. In anechoic chamber and reverberation environment, the algorithm proposed in this paper is compared and analyzed from three evaluation indexes , interaural time difference , interaural level difference and interaural cross-correlation. The experimental results show that the binaural rendering algorithm proposed in this paper can achieve better results under different reverberation environments and sound source directions, which is superior to the existing algorithms. The performance of open and rigid spherical array under different binaural rendering algorithms is compared to verify the superiority of rigid ball.
	
	
	\begin{Keywords}
		Binaural rendering, head-related transfer function, spatial audio, spherical harmonic decomposition, sound field amplification
	\end{Keywords}
\end{Abstract}%中英文摘要
\begin{innovation}
	
(1)这是一个创新点
\end{innovation}%创新点
\tableofcontents
%
\mainmatter  % 开始各章节%如果此处include无法成功,将子tex文件的Document的Document setting的 CPconvert的Document Format更改为UTF-8
\include{data/chapter1}
\include{data/chapter2}
\chapter{更新信息}
\section{更新历史}
材料学院于2022年12月7日已通知:从12月5日开始,启用新论文模板,该模板请各位同学慎重使用,如有时间,我会在近期对该模板进行更新。

2023/02/09 更新了论文封面部分,其他部分差距不大,可以自行更改或联系我标明

2023/11/04 更新了论文细节,参考文献格式,增加对双语图片和表格标题的支持

2024/01/28 交完论文后,汇总了写论文期间的改动

\section{已知问题}
限于\LaTeX 自身特性和作者水平,目前此模板还存在如下问题:

宋体黑体加粗表现与word存在些许不一致

章节标题中,章节编号数字及英文不应加粗 (已解决 2024.02.25)

图片/表格与正文间距不合适,可根据文档自行更改模板 (已解决 2024.02.25)

精力问题,部分页为后添加,无法完成自动编号,可以自行更改

声明页疑似与模板中字体大小不一致,但学校模板本身存在冲突
\chapter{实验}

\begin{lrbox}{\mysavebox}%
	% generated by Plantuml 1.2024.7       
\definecolor{plantucolor0000}{RGB}{24,24,24}
\definecolor{plantucolor0001}{RGB}{0,0,0}
\definecolor{plantucolor0002}{RGB}{226,226,240}
\scalebox{0.63}{
	\begin{tikzpicture}[yscale=-1
		,pstyle0/.style={color=plantucolor0000,line width=0.5pt,dash pattern=on 5.0pt off 5.0pt}
		,pstyle1/.style={color=plantucolor0000,fill=plantucolor0002,line width=0.5pt}
		,pstyle2/.style={color=plantucolor0000,line width=0.5pt}
		,pstyle3/.style={color=plantucolor0000,fill=plantucolor0000,line width=1.0pt}
		,pstyle4/.style={color=plantucolor0000,line width=1.0pt}
		,pstyle5/.style={color=plantucolor0000,line width=1.0pt,dash pattern=on 2.0pt off 2.0pt}
		]
		\draw[pstyle0] (41pt,83.6211pt) -- (41pt,2606.9023pt);
		\draw[pstyle0] (108.8634pt,83.6211pt) -- (108.8634pt,2606.9023pt);
		\draw[pstyle0] (139.9777pt,83.6211pt) -- (139.9777pt,2606.9023pt);
		\draw[pstyle0] (176.1807pt,83.6211pt) -- (176.1807pt,2606.9023pt);
		\draw[pstyle0] (211.4093pt,83.6211pt) -- (211.4093pt,2606.9023pt);
		\draw[pstyle0] (246.6123pt,83.6211pt) -- (246.6123pt,2606.9023pt);
		\draw[pstyle0] (285.9296pt,83.6211pt) -- (285.9296pt,2606.9023pt);
		\draw[pstyle0] (325.2469pt,83.6211pt) -- (325.2469pt,2606.9023pt);
		\draw[pstyle0] (365.5642pt,83.6211pt) -- (365.5642pt,2606.9023pt);
		\draw[pstyle0] (405.7928pt,83.6211pt) -- (405.7928pt,2606.9023pt);
		\draw[pstyle0] (440.0214pt,83.6211pt) -- (440.0214pt,2606.9023pt);
		\draw[pstyle0] (472.4256pt,83.6211pt) -- (472.4256pt,2606.9023pt);
		\draw[pstyle0] (514.2341pt,83.6211pt) -- (514.2341pt,2606.9023pt);
		\draw[pstyle0] (566.5769pt,83.6211pt) -- (566.5769pt,2606.9023pt);
		\draw[pstyle0] (617.0341pt,83.6211pt) -- (617.0341pt,2606.9023pt);
		\draw[pstyle0] (672.651pt,83.6211pt) -- (672.651pt,2606.9023pt);
		\node at (5pt,65pt)[below right,color=black]{DISSPUD};
		\draw[pstyle1] (41.8201pt,13.5pt) ellipse (8pt and 8pt);
		\draw[pstyle2] (41.8201pt,21.5pt) -- (41.8201pt,48.5pt)(28.8201pt,29.5pt) -- (54.8201pt,29.5pt)(41.8201pt,48.5pt) -- (28.8201pt,63.5pt)(41.8201pt,48.5pt) -- (54.8201pt,63.5pt);
		\node at (5pt,2605.9023pt)[below right,color=black]{DISSPUD};
		\draw[pstyle1] (41.8201pt,2633.0234pt) ellipse (8pt and 8pt);
		\draw[pstyle2] (41.8201pt,2641.0234pt) -- (41.8201pt,2668.0234pt)(28.8201pt,2649.0234pt) -- (54.8201pt,2649.0234pt)(41.8201pt,2668.0234pt) -- (28.8201pt,2683.0234pt)(41.8201pt,2668.0234pt) -- (54.8201pt,2683.0234pt);
		\draw[pstyle1] (97.8634pt,55pt) arc (180:270:5pt) -- (102.8634pt,50pt) -- (114.9777pt,50pt) arc (270:360:5pt) -- (119.9777pt,55pt) -- (119.9777pt,77.6211pt) arc (0:90:5pt) -- (114.9777pt,82.6211pt) -- (102.8634pt,82.6211pt) arc (90:180:5pt) -- (97.8634pt,77.6211pt) -- cycle;
		\node at (104.8634pt,57pt)[below right,color=black]{b};
		\draw[pstyle1] (97.8634pt,2610.9023pt) arc (180:270:5pt) -- (102.8634pt,2605.9023pt) -- (114.9777pt,2605.9023pt) arc (270:360:5pt) -- (119.9777pt,2610.9023pt) -- (119.9777pt,2633.5234pt) arc (0:90:5pt) -- (114.9777pt,2638.5234pt) -- (102.8634pt,2638.5234pt) arc (90:180:5pt) -- (97.8634pt,2633.5234pt) -- cycle;
		\node at (104.8634pt,2612.9023pt)[below right,color=black]{b};
		\draw[pstyle1] (129.9777pt,55pt) arc (180:270:5pt) -- (134.9777pt,50pt) -- (146.1807pt,50pt) arc (270:360:5pt) -- (151.1807pt,55pt) -- (151.1807pt,77.6211pt) arc (0:90:5pt) -- (146.1807pt,82.6211pt) -- (134.9777pt,82.6211pt) arc (90:180:5pt) -- (129.9777pt,77.6211pt) -- cycle;
		\node at (136.9777pt,57pt)[below right,color=black]{c};
		\draw[pstyle1] (129.9777pt,2610.9023pt) arc (180:270:5pt) -- (134.9777pt,2605.9023pt) -- (146.1807pt,2605.9023pt) arc (270:360:5pt) -- (151.1807pt,2610.9023pt) -- (151.1807pt,2633.5234pt) arc (0:90:5pt) -- (146.1807pt,2638.5234pt) -- (134.9777pt,2638.5234pt) arc (90:180:5pt) -- (129.9777pt,2633.5234pt) -- cycle;
		\node at (136.9777pt,2612.9023pt)[below right,color=black]{c};
		\draw[pstyle1] (161.1807pt,55pt) arc (180:270:5pt) -- (166.1807pt,50pt) -- (186.4093pt,50pt) arc (270:360:5pt) -- (191.4093pt,55pt) -- (191.4093pt,77.6211pt) arc (0:90:5pt) -- (186.4093pt,82.6211pt) -- (166.1807pt,82.6211pt) arc (90:180:5pt) -- (161.1807pt,77.6211pt) -- cycle;
		\node at (168.1807pt,57pt)[below right,color=black]{db};
		\draw[pstyle1] (161.1807pt,2610.9023pt) arc (180:270:5pt) -- (166.1807pt,2605.9023pt) -- (186.4093pt,2605.9023pt) arc (270:360:5pt) -- (191.4093pt,2610.9023pt) -- (191.4093pt,2633.5234pt) arc (0:90:5pt) -- (186.4093pt,2638.5234pt) -- (166.1807pt,2638.5234pt) arc (90:180:5pt) -- (161.1807pt,2633.5234pt) -- cycle;
		\node at (168.1807pt,2612.9023pt)[below right,color=black]{db};
		\draw[pstyle1] (201.4093pt,55pt) arc (180:270:5pt) -- (206.4093pt,50pt) -- (217.6123pt,50pt) arc (270:360:5pt) -- (222.6123pt,55pt) -- (222.6123pt,77.6211pt) arc (0:90:5pt) -- (217.6123pt,82.6211pt) -- (206.4093pt,82.6211pt) arc (90:180:5pt) -- (201.4093pt,77.6211pt) -- cycle;
		\node at (208.4093pt,57pt)[below right,color=black]{e};
		\draw[pstyle1] (201.4093pt,2610.9023pt) arc (180:270:5pt) -- (206.4093pt,2605.9023pt) -- (217.6123pt,2605.9023pt) arc (270:360:5pt) -- (222.6123pt,2610.9023pt) -- (222.6123pt,2633.5234pt) arc (0:90:5pt) -- (217.6123pt,2638.5234pt) -- (206.4093pt,2638.5234pt) arc (90:180:5pt) -- (201.4093pt,2633.5234pt) -- cycle;
		\node at (208.4093pt,2612.9023pt)[below right,color=black]{e};
		\draw[pstyle1] (232.6123pt,55pt) arc (180:270:5pt) -- (237.6123pt,50pt) -- (256.9296pt,50pt) arc (270:360:5pt) -- (261.9296pt,55pt) -- (261.9296pt,77.6211pt) arc (0:90:5pt) -- (256.9296pt,82.6211pt) -- (237.6123pt,82.6211pt) arc (90:180:5pt) -- (232.6123pt,77.6211pt) -- cycle;
		\node at (239.6123pt,57pt)[below right,color=black]{bz};
		\draw[pstyle1] (232.6123pt,2610.9023pt) arc (180:270:5pt) -- (237.6123pt,2605.9023pt) -- (256.9296pt,2605.9023pt) arc (270:360:5pt) -- (261.9296pt,2610.9023pt) -- (261.9296pt,2633.5234pt) arc (0:90:5pt) -- (256.9296pt,2638.5234pt) -- (237.6123pt,2638.5234pt) arc (90:180:5pt) -- (232.6123pt,2633.5234pt) -- cycle;
		\node at (239.6123pt,2612.9023pt)[below right,color=black]{bz};
		\draw[pstyle1] (271.9296pt,55pt) arc (180:270:5pt) -- (276.9296pt,50pt) -- (296.2469pt,50pt) arc (270:360:5pt) -- (301.2469pt,55pt) -- (301.2469pt,77.6211pt) arc (0:90:5pt) -- (296.2469pt,82.6211pt) -- (276.9296pt,82.6211pt) arc (90:180:5pt) -- (271.9296pt,77.6211pt) -- cycle;
		\node at (278.9296pt,57pt)[below right,color=black]{bc};
		\draw[pstyle1] (271.9296pt,2610.9023pt) arc (180:270:5pt) -- (276.9296pt,2605.9023pt) -- (296.2469pt,2605.9023pt) arc (270:360:5pt) -- (301.2469pt,2610.9023pt) -- (301.2469pt,2633.5234pt) arc (0:90:5pt) -- (296.2469pt,2638.5234pt) -- (276.9296pt,2638.5234pt) arc (90:180:5pt) -- (271.9296pt,2633.5234pt) -- cycle;
		\node at (278.9296pt,2612.9023pt)[below right,color=black]{bc};
		\draw[pstyle1] (311.2469pt,55pt) arc (180:270:5pt) -- (316.2469pt,50pt) -- (335.5642pt,50pt) arc (270:360:5pt) -- (340.5642pt,55pt) -- (340.5642pt,77.6211pt) arc (0:90:5pt) -- (335.5642pt,82.6211pt) -- (316.2469pt,82.6211pt) arc (90:180:5pt) -- (311.2469pt,77.6211pt) -- cycle;
		\node at (318.2469pt,57pt)[below right,color=black]{ab};
		\draw[pstyle1] (311.2469pt,2610.9023pt) arc (180:270:5pt) -- (316.2469pt,2605.9023pt) -- (335.5642pt,2605.9023pt) arc (270:360:5pt) -- (340.5642pt,2610.9023pt) -- (340.5642pt,2633.5234pt) arc (0:90:5pt) -- (335.5642pt,2638.5234pt) -- (316.2469pt,2638.5234pt) arc (90:180:5pt) -- (311.2469pt,2633.5234pt) -- cycle;
		\node at (318.2469pt,2612.9023pt)[below right,color=black]{ab};
		\draw[pstyle1] (350.5642pt,55pt) arc (180:270:5pt) -- (355.5642pt,50pt) -- (375.7928pt,50pt) arc (270:360:5pt) -- (380.7928pt,55pt) -- (380.7928pt,77.6211pt) arc (0:90:5pt) -- (375.7928pt,82.6211pt) -- (355.5642pt,82.6211pt) arc (90:180:5pt) -- (350.5642pt,77.6211pt) -- cycle;
		\node at (357.5642pt,57pt)[below right,color=black]{bg};
		\draw[pstyle1] (350.5642pt,2610.9023pt) arc (180:270:5pt) -- (355.5642pt,2605.9023pt) -- (375.7928pt,2605.9023pt) arc (270:360:5pt) -- (380.7928pt,2610.9023pt) -- (380.7928pt,2633.5234pt) arc (0:90:5pt) -- (375.7928pt,2638.5234pt) -- (355.5642pt,2638.5234pt) arc (90:180:5pt) -- (350.5642pt,2633.5234pt) -- cycle;
		\node at (357.5642pt,2612.9023pt)[below right,color=black]{bg};
		\draw[pstyle1] (390.7928pt,55pt) arc (180:270:5pt) -- (395.7928pt,50pt) -- (416.0214pt,50pt) arc (270:360:5pt) -- (421.0214pt,55pt) -- (421.0214pt,77.6211pt) arc (0:90:5pt) -- (416.0214pt,82.6211pt) -- (395.7928pt,82.6211pt) arc (90:180:5pt) -- (390.7928pt,77.6211pt) -- cycle;
		\node at (397.7928pt,57pt)[below right,color=black]{bh};
		\draw[pstyle1] (390.7928pt,2610.9023pt) arc (180:270:5pt) -- (395.7928pt,2605.9023pt) -- (416.0214pt,2605.9023pt) arc (270:360:5pt) -- (421.0214pt,2610.9023pt) -- (421.0214pt,2633.5234pt) arc (0:90:5pt) -- (416.0214pt,2638.5234pt) -- (395.7928pt,2638.5234pt) arc (90:180:5pt) -- (390.7928pt,2633.5234pt) -- cycle;
		\node at (397.7928pt,2612.9023pt)[below right,color=black]{bh};
		\draw[pstyle1] (431.0214pt,55pt) arc (180:270:5pt) -- (436.0214pt,50pt) -- (445.4256pt,50pt) arc (270:360:5pt) -- (450.4256pt,55pt) -- (450.4256pt,77.6211pt) arc (0:90:5pt) -- (445.4256pt,82.6211pt) -- (436.0214pt,82.6211pt) arc (90:180:5pt) -- (431.0214pt,77.6211pt) -- cycle;
		\node at (438.0214pt,57pt)[below right,color=black]{f};
		\draw[pstyle1] (431.0214pt,2610.9023pt) arc (180:270:5pt) -- (436.0214pt,2605.9023pt) -- (445.4256pt,2605.9023pt) arc (270:360:5pt) -- (450.4256pt,2610.9023pt) -- (450.4256pt,2633.5234pt) arc (0:90:5pt) -- (445.4256pt,2638.5234pt) -- (436.0214pt,2638.5234pt) arc (90:180:5pt) -- (431.0214pt,2633.5234pt) -- cycle;
		\node at (438.0214pt,2612.9023pt)[below right,color=black]{f};
		\draw[pstyle1] (460.4256pt,55pt) arc (180:270:5pt) -- (465.4256pt,50pt) -- (480.2341pt,50pt) arc (270:360:5pt) -- (485.2341pt,55pt) -- (485.2341pt,77.6211pt) arc (0:90:5pt) -- (480.2341pt,82.6211pt) -- (465.4256pt,82.6211pt) arc (90:180:5pt) -- (460.4256pt,77.6211pt) -- cycle;
		\node at (467.4256pt,57pt)[below right,color=black]{ff};
		\draw[pstyle1] (460.4256pt,2610.9023pt) arc (180:270:5pt) -- (465.4256pt,2605.9023pt) -- (480.2341pt,2605.9023pt) arc (270:360:5pt) -- (485.2341pt,2610.9023pt) -- (485.2341pt,2633.5234pt) arc (0:90:5pt) -- (480.2341pt,2638.5234pt) -- (465.4256pt,2638.5234pt) arc (90:180:5pt) -- (460.4256pt,2633.5234pt) -- cycle;
		\node at (467.4256pt,2612.9023pt)[below right,color=black]{ff};
		\draw[pstyle1] (495.2341pt,55pt) arc (180:270:5pt) -- (500.2341pt,50pt) -- (528.5769pt,50pt) arc (270:360:5pt) -- (533.5769pt,55pt) -- (533.5769pt,77.6211pt) arc (0:90:5pt) -- (528.5769pt,82.6211pt) -- (500.2341pt,82.6211pt) arc (90:180:5pt) -- (495.2341pt,77.6211pt) -- cycle;
		\node at (502.2341pt,57pt)[below right,color=black]{ggg};
		\draw[pstyle1] (495.2341pt,2610.9023pt) arc (180:270:5pt) -- (500.2341pt,2605.9023pt) -- (528.5769pt,2605.9023pt) arc (270:360:5pt) -- (533.5769pt,2610.9023pt) -- (533.5769pt,2633.5234pt) arc (0:90:5pt) -- (528.5769pt,2638.5234pt) -- (500.2341pt,2638.5234pt) arc (90:180:5pt) -- (495.2341pt,2633.5234pt) -- cycle;
		\node at (502.2341pt,2612.9023pt)[below right,color=black]{ggg};
		\draw[pstyle1] (543.5769pt,55pt) arc (180:270:5pt) -- (548.5769pt,50pt) -- (585.0341pt,50pt) arc (270:360:5pt) -- (590.0341pt,55pt) -- (590.0341pt,77.6211pt) arc (0:90:5pt) -- (585.0341pt,82.6211pt) -- (548.5769pt,82.6211pt) arc (90:180:5pt) -- (543.5769pt,77.6211pt) -- cycle;
		\node at (550.5769pt,57pt)[below right,color=black]{vvvv};
		\draw[pstyle1] (543.5769pt,2610.9023pt) arc (180:270:5pt) -- (548.5769pt,2605.9023pt) -- (585.0341pt,2605.9023pt) arc (270:360:5pt) -- (590.0341pt,2610.9023pt) -- (590.0341pt,2633.5234pt) arc (0:90:5pt) -- (585.0341pt,2638.5234pt) -- (548.5769pt,2638.5234pt) arc (90:180:5pt) -- (543.5769pt,2633.5234pt) -- cycle;
		\node at (550.5769pt,2612.9023pt)[below right,color=black]{vvvv};
		\draw[pstyle1] (600.0341pt,55pt) arc (180:270:5pt) -- (605.0341pt,50pt) -- (630.651pt,50pt) arc (270:360:5pt) -- (635.651pt,55pt) -- (635.651pt,77.6211pt) arc (0:90:5pt) -- (630.651pt,82.6211pt) -- (605.0341pt,82.6211pt) arc (90:180:5pt) -- (600.0341pt,77.6211pt) -- cycle;
		\node at (607.0341pt,57pt)[below right,color=black]{ffff};
		\draw[pstyle1] (600.0341pt,2610.9023pt) arc (180:270:5pt) -- (605.0341pt,2605.9023pt) -- (630.651pt,2605.9023pt) arc (270:360:5pt) -- (635.651pt,2610.9023pt) -- (635.651pt,2633.5234pt) arc (0:90:5pt) -- (630.651pt,2638.5234pt) -- (605.0341pt,2638.5234pt) arc (90:180:5pt) -- (600.0341pt,2633.5234pt) -- cycle;
		\node at (607.0341pt,2612.9023pt)[below right,color=black]{ffff};
		\draw[pstyle1] (645.651pt,55pt) arc (180:270:5pt) -- (650.651pt,50pt) -- (695.2225pt,50pt) arc (270:360:5pt) -- (700.2225pt,55pt) -- (700.2225pt,77.6211pt) arc (0:90:5pt) -- (695.2225pt,82.6211pt) -- (650.651pt,82.6211pt) arc (90:180:5pt) -- (645.651pt,77.6211pt) -- cycle;
		\node at (652.651pt,57pt)[below right,color=black]{bgggg};
		\draw[pstyle1] (645.651pt,2610.9023pt) arc (180:270:5pt) -- (650.651pt,2605.9023pt) -- (695.2225pt,2605.9023pt) arc (270:360:5pt) -- (700.2225pt,2610.9023pt) -- (700.2225pt,2633.5234pt) arc (0:90:5pt) -- (695.2225pt,2638.5234pt) -- (650.651pt,2638.5234pt) arc (90:180:5pt) -- (645.651pt,2633.5234pt) -- cycle;
		\node at (652.651pt,2612.9023pt)[below right,color=black]{bgggg};
		\draw[pstyle3] (96.9206pt,112.9121pt) -- (106.9206pt,116.9121pt) -- (96.9206pt,120.9121pt) -- (100.9206pt,116.9121pt) -- cycle;
		\draw[pstyle4] (41.8201pt,116.9121pt) -- (102.9206pt,116.9121pt);
		\node at (48.8201pt,97.6211pt)[below right,color=black]{\textbf{1}};
		\node at (60.3124pt,97.6211pt)[below right,color=black]{aaa};
		\draw[pstyle3] (52.8201pt,144.2031pt) -- (42.8201pt,148.2031pt) -- (52.8201pt,152.2031pt) -- (48.8201pt,148.2031pt) -- cycle;
		\draw[pstyle5] (46.8201pt,148.2031pt) -- (107.9206pt,148.2031pt);
		\node at (58.8201pt,128.9121pt)[below right,color=black]{\textbf{2}};
		\node at (70.3124pt,128.9121pt)[below right,color=black]{:aaa};
		\draw[pstyle3] (128.5792pt,175.4941pt) -- (138.5792pt,179.4941pt) -- (128.5792pt,183.4941pt) -- (132.5792pt,179.4941pt) -- cycle;
		\draw[pstyle4] (41.8201pt,179.4941pt) -- (134.5792pt,179.4941pt);
		\node at (48.8201pt,160.2031pt)[below right,color=black]{\textbf{3}};
		\node at (60.3124pt,160.2031pt)[below right,color=black]{aaa};
		\draw[pstyle3] (52.8201pt,206.7852pt) -- (42.8201pt,210.7852pt) -- (52.8201pt,214.7852pt) -- (48.8201pt,210.7852pt) -- cycle;
		\draw[pstyle5] (46.8201pt,210.7852pt) -- (139.5792pt,210.7852pt);
		\node at (58.8201pt,191.4941pt)[below right,color=black]{\textbf{4}};
		\node at (70.3124pt,191.4941pt)[below right,color=black]{:aaa};
		\draw[pstyle3] (164.295pt,238.0762pt) -- (174.295pt,242.0762pt) -- (164.295pt,246.0762pt) -- (168.295pt,242.0762pt) -- cycle;
		\draw[pstyle4] (41.8201pt,242.0762pt) -- (170.295pt,242.0762pt);
		\node at (48.8201pt,222.7852pt)[below right,color=black]{\textbf{5}};
		\node at (60.3124pt,222.7852pt)[below right,color=black]{aaa};
		\draw[pstyle3] (52.8201pt,269.3672pt) -- (42.8201pt,273.3672pt) -- (52.8201pt,277.3672pt) -- (48.8201pt,273.3672pt) -- cycle;
		\draw[pstyle5] (46.8201pt,273.3672pt) -- (175.295pt,273.3672pt);
		\node at (58.8201pt,254.0762pt)[below right,color=black]{\textbf{6}};
		\node at (70.3124pt,254.0762pt)[below right,color=black]{:aaa};
		\draw[pstyle3] (200.0108pt,300.6582pt) -- (210.0108pt,304.6582pt) -- (200.0108pt,308.6582pt) -- (204.0108pt,304.6582pt) -- cycle;
		\draw[pstyle4] (41.8201pt,304.6582pt) -- (206.0108pt,304.6582pt);
		\node at (48.8201pt,285.3672pt)[below right,color=black]{\textbf{7}};
		\node at (60.3124pt,285.3672pt)[below right,color=black]{aaa};
		\draw[pstyle3] (52.8201pt,331.9492pt) -- (42.8201pt,335.9492pt) -- (52.8201pt,339.9492pt) -- (48.8201pt,335.9492pt) -- cycle;
		\draw[pstyle5] (46.8201pt,335.9492pt) -- (211.0108pt,335.9492pt);
		\node at (58.8201pt,316.6582pt)[below right,color=black]{\textbf{8}};
		\node at (70.3124pt,316.6582pt)[below right,color=black]{:aaa};
		\draw[pstyle3] (96.9206pt,363.2402pt) -- (106.9206pt,367.2402pt) -- (96.9206pt,371.2402pt) -- (100.9206pt,367.2402pt) -- cycle;
		\draw[pstyle4] (41.8201pt,367.2402pt) -- (102.9206pt,367.2402pt);
		\node at (48.8201pt,347.9492pt)[below right,color=black]{\textbf{9}};
		\node at (60.3124pt,347.9492pt)[below right,color=black]{aaa};
		\draw[pstyle3] (52.8201pt,394.5313pt) -- (42.8201pt,398.5313pt) -- (52.8201pt,402.5313pt) -- (48.8201pt,398.5313pt) -- cycle;
		\draw[pstyle5] (46.8201pt,398.5313pt) -- (107.9206pt,398.5313pt);
		\node at (58.8201pt,379.2402pt)[below right,color=black]{\textbf{10}};
		\node at (77.8047pt,379.2402pt)[below right,color=black]{:aaa};
		\draw[pstyle3] (96.9206pt,425.8223pt) -- (106.9206pt,429.8223pt) -- (96.9206pt,433.8223pt) -- (100.9206pt,429.8223pt) -- cycle;
		\draw[pstyle4] (41.8201pt,429.8223pt) -- (102.9206pt,429.8223pt);
		\node at (48.8201pt,410.5313pt)[below right,color=black]{\textbf{11}};
		\node at (67.8047pt,410.5313pt)[below right,color=black]{aaa};
		\draw[pstyle3] (52.8201pt,457.1133pt) -- (42.8201pt,461.1133pt) -- (52.8201pt,465.1133pt) -- (48.8201pt,461.1133pt) -- cycle;
		\draw[pstyle5] (46.8201pt,461.1133pt) -- (107.9206pt,461.1133pt);
		\node at (58.8201pt,441.8223pt)[below right,color=black]{\textbf{12}};
		\node at (77.8047pt,441.8223pt)[below right,color=black]{:aaa};
		\draw[pstyle3] (96.9206pt,488.4043pt) -- (106.9206pt,492.4043pt) -- (96.9206pt,496.4043pt) -- (100.9206pt,492.4043pt) -- cycle;
		\draw[pstyle4] (41.8201pt,492.4043pt) -- (102.9206pt,492.4043pt);
		\node at (48.8201pt,473.1133pt)[below right,color=black]{\textbf{13}};
		\node at (67.8047pt,473.1133pt)[below right,color=black]{aaa};
		\draw[pstyle3] (52.8201pt,519.6953pt) -- (42.8201pt,523.6953pt) -- (52.8201pt,527.6953pt) -- (48.8201pt,523.6953pt) -- cycle;
		\draw[pstyle5] (46.8201pt,523.6953pt) -- (107.9206pt,523.6953pt);
		\node at (58.8201pt,504.4043pt)[below right,color=black]{\textbf{14}};
		\node at (77.8047pt,504.4043pt)[below right,color=black]{:aaa};
		\draw[pstyle3] (235.271pt,550.9863pt) -- (245.271pt,554.9863pt) -- (235.271pt,558.9863pt) -- (239.271pt,554.9863pt) -- cycle;
		\draw[pstyle4] (41.8201pt,554.9863pt) -- (241.271pt,554.9863pt);
		\node at (48.8201pt,535.6953pt)[below right,color=black]{\textbf{15}};
		\node at (67.8047pt,535.6953pt)[below right,color=black]{aaa};
		\draw[pstyle3] (52.8201pt,582.2773pt) -- (42.8201pt,586.2773pt) -- (52.8201pt,590.2773pt) -- (48.8201pt,586.2773pt) -- cycle;
		\draw[pstyle5] (46.8201pt,586.2773pt) -- (246.271pt,586.2773pt);
		\node at (58.8201pt,566.9863pt)[below right,color=black]{\textbf{16}};
		\node at (77.8047pt,566.9863pt)[below right,color=black]{:aaa};
		\draw[pstyle3] (274.5883pt,613.5684pt) -- (284.5883pt,617.5684pt) -- (274.5883pt,621.5684pt) -- (278.5883pt,617.5684pt) -- cycle;
		\draw[pstyle4] (41.8201pt,617.5684pt) -- (280.5883pt,617.5684pt);
		\node at (48.8201pt,598.2773pt)[below right,color=black]{\textbf{17}};
		\node at (67.8047pt,598.2773pt)[below right,color=black]{aaa};
		\draw[pstyle3] (52.8201pt,644.8594pt) -- (42.8201pt,648.8594pt) -- (52.8201pt,652.8594pt) -- (48.8201pt,648.8594pt) -- cycle;
		\draw[pstyle5] (46.8201pt,648.8594pt) -- (285.5883pt,648.8594pt);
		\node at (58.8201pt,629.5684pt)[below right,color=black]{\textbf{18}};
		\node at (77.8047pt,629.5684pt)[below right,color=black]{:aaa};
		\draw[pstyle3] (313.9056pt,676.1504pt) -- (323.9056pt,680.1504pt) -- (313.9056pt,684.1504pt) -- (317.9056pt,680.1504pt) -- cycle;
		\draw[pstyle4] (41.8201pt,680.1504pt) -- (319.9056pt,680.1504pt);
		\node at (48.8201pt,660.8594pt)[below right,color=black]{\textbf{19}};
		\node at (67.8047pt,660.8594pt)[below right,color=black]{aaa};
		\draw[pstyle3] (52.8201pt,707.4414pt) -- (42.8201pt,711.4414pt) -- (52.8201pt,715.4414pt) -- (48.8201pt,711.4414pt) -- cycle;
		\draw[pstyle5] (46.8201pt,711.4414pt) -- (324.9056pt,711.4414pt);
		\node at (58.8201pt,692.1504pt)[below right,color=black]{\textbf{20}};
		\node at (77.8047pt,692.1504pt)[below right,color=black]{:aaa};
		\draw[pstyle3] (96.9206pt,738.7324pt) -- (106.9206pt,742.7324pt) -- (96.9206pt,746.7324pt) -- (100.9206pt,742.7324pt) -- cycle;
		\draw[pstyle4] (41.8201pt,742.7324pt) -- (102.9206pt,742.7324pt);
		\node at (48.8201pt,723.4414pt)[below right,color=black]{\textbf{21}};
		\node at (67.8047pt,723.4414pt)[below right,color=black]{aaa};
		\draw[pstyle3] (52.8201pt,770.0234pt) -- (42.8201pt,774.0234pt) -- (52.8201pt,778.0234pt) -- (48.8201pt,774.0234pt) -- cycle;
		\draw[pstyle5] (46.8201pt,774.0234pt) -- (107.9206pt,774.0234pt);
		\node at (58.8201pt,754.7324pt)[below right,color=black]{\textbf{22}};
		\node at (77.8047pt,754.7324pt)[below right,color=black]{:aaa};
		\draw[pstyle3] (353.6785pt,801.3145pt) -- (363.6785pt,805.3145pt) -- (353.6785pt,809.3145pt) -- (357.6785pt,805.3145pt) -- cycle;
		\draw[pstyle4] (41.8201pt,805.3145pt) -- (359.6785pt,805.3145pt);
		\node at (48.8201pt,786.0234pt)[below right,color=black]{\textbf{23}};
		\node at (67.8047pt,786.0234pt)[below right,color=black]{aaa};
		\draw[pstyle3] (52.8201pt,832.6055pt) -- (42.8201pt,836.6055pt) -- (52.8201pt,840.6055pt) -- (48.8201pt,836.6055pt) -- cycle;
		\draw[pstyle5] (46.8201pt,836.6055pt) -- (364.6785pt,836.6055pt);
		\node at (58.8201pt,817.3145pt)[below right,color=black]{\textbf{24}};
		\node at (77.8047pt,817.3145pt)[below right,color=black]{:aaa};
		\draw[pstyle3] (393.9071pt,863.8965pt) -- (403.9071pt,867.8965pt) -- (393.9071pt,871.8965pt) -- (397.9071pt,867.8965pt) -- cycle;
		\draw[pstyle4] (41.8201pt,867.8965pt) -- (399.9071pt,867.8965pt);
		\node at (48.8201pt,848.6055pt)[below right,color=black]{\textbf{25}};
		\node at (67.8047pt,848.6055pt)[below right,color=black]{aaa};
		\draw[pstyle3] (52.8201pt,895.1875pt) -- (42.8201pt,899.1875pt) -- (52.8201pt,903.1875pt) -- (48.8201pt,899.1875pt) -- cycle;
		\draw[pstyle5] (46.8201pt,899.1875pt) -- (404.9071pt,899.1875pt);
		\node at (58.8201pt,879.8965pt)[below right,color=black]{\textbf{26}};
		\node at (77.8047pt,879.8965pt)[below right,color=black]{:aaa};
		\draw[pstyle3] (96.9206pt,926.4785pt) -- (106.9206pt,930.4785pt) -- (96.9206pt,934.4785pt) -- (100.9206pt,930.4785pt) -- cycle;
		\draw[pstyle4] (41.8201pt,930.4785pt) -- (102.9206pt,930.4785pt);
		\node at (48.8201pt,911.1875pt)[below right,color=black]{\textbf{27}};
		\node at (67.8047pt,911.1875pt)[below right,color=black]{aaa};
		\draw[pstyle3] (52.8201pt,957.7695pt) -- (42.8201pt,961.7695pt) -- (52.8201pt,965.7695pt) -- (48.8201pt,961.7695pt) -- cycle;
		\draw[pstyle5] (46.8201pt,961.7695pt) -- (107.9206pt,961.7695pt);
		\node at (58.8201pt,942.4785pt)[below right,color=black]{\textbf{28}};
		\node at (77.8047pt,942.4785pt)[below right,color=black]{:aaa};
		\draw[pstyle3] (96.9206pt,989.0605pt) -- (106.9206pt,993.0605pt) -- (96.9206pt,997.0605pt) -- (100.9206pt,993.0605pt) -- cycle;
		\draw[pstyle4] (41.8201pt,993.0605pt) -- (102.9206pt,993.0605pt);
		\node at (48.8201pt,973.7695pt)[below right,color=black]{\textbf{29}};
		\node at (67.8047pt,973.7695pt)[below right,color=black]{aaa};
		\draw[pstyle3] (52.8201pt,1020.3516pt) -- (42.8201pt,1024.3516pt) -- (52.8201pt,1028.3516pt) -- (48.8201pt,1024.3516pt) -- cycle;
		\draw[pstyle5] (46.8201pt,1024.3516pt) -- (107.9206pt,1024.3516pt);
		\node at (58.8201pt,1005.0605pt)[below right,color=black]{\textbf{30}};
		\node at (77.8047pt,1005.0605pt)[below right,color=black]{:aaa};
		\draw[pstyle3] (96.9206pt,1051.6426pt) -- (106.9206pt,1055.6426pt) -- (96.9206pt,1059.6426pt) -- (100.9206pt,1055.6426pt) -- cycle;
		\draw[pstyle4] (41.8201pt,1055.6426pt) -- (102.9206pt,1055.6426pt);
		\node at (48.8201pt,1036.3516pt)[below right,color=black]{\textbf{31}};
		\node at (67.8047pt,1036.3516pt)[below right,color=black]{aaa};
		\draw[pstyle3] (52.8201pt,1082.9336pt) -- (42.8201pt,1086.9336pt) -- (52.8201pt,1090.9336pt) -- (48.8201pt,1086.9336pt) -- cycle;
		\draw[pstyle5] (46.8201pt,1086.9336pt) -- (107.9206pt,1086.9336pt);
		\node at (58.8201pt,1067.6426pt)[below right,color=black]{\textbf{32}};
		\node at (77.8047pt,1067.6426pt)[below right,color=black]{:aaa};
		\draw[pstyle3] (428.7235pt,1114.2246pt) -- (438.7235pt,1118.2246pt) -- (428.7235pt,1122.2246pt) -- (432.7235pt,1118.2246pt) -- cycle;
		\draw[pstyle4] (41.8201pt,1118.2246pt) -- (434.7235pt,1118.2246pt);
		\node at (48.8201pt,1098.9336pt)[below right,color=black]{\textbf{33}};
		\node at (67.8047pt,1098.9336pt)[below right,color=black]{aaa};
		\draw[pstyle3] (52.8201pt,1145.5156pt) -- (42.8201pt,1149.5156pt) -- (52.8201pt,1153.5156pt) -- (48.8201pt,1149.5156pt) -- cycle;
		\draw[pstyle5] (46.8201pt,1149.5156pt) -- (439.7235pt,1149.5156pt);
		\node at (58.8201pt,1130.2246pt)[below right,color=black]{\textbf{34}};
		\node at (77.8047pt,1130.2246pt)[below right,color=black]{:aaa};
		\draw[pstyle3] (460.8298pt,1176.8066pt) -- (470.8298pt,1180.8066pt) -- (460.8298pt,1184.8066pt) -- (464.8298pt,1180.8066pt) -- cycle;
		\draw[pstyle4] (41.8201pt,1180.8066pt) -- (466.8298pt,1180.8066pt);
		\node at (48.8201pt,1161.5156pt)[below right,color=black]{\textbf{35}};
		\node at (67.8047pt,1161.5156pt)[below right,color=black]{aaa};
		\draw[pstyle3] (52.8201pt,1208.0977pt) -- (42.8201pt,1212.0977pt) -- (52.8201pt,1216.0977pt) -- (48.8201pt,1212.0977pt) -- cycle;
		\draw[pstyle5] (46.8201pt,1212.0977pt) -- (471.8298pt,1212.0977pt);
		\node at (58.8201pt,1192.8066pt)[below right,color=black]{\textbf{36}};
		\node at (77.8047pt,1192.8066pt)[below right,color=black]{:aaa};
		\draw[pstyle3] (502.4055pt,1239.3887pt) -- (512.4055pt,1243.3887pt) -- (502.4055pt,1247.3887pt) -- (506.4055pt,1243.3887pt) -- cycle;
		\draw[pstyle4] (41.8201pt,1243.3887pt) -- (508.4055pt,1243.3887pt);
		\node at (48.8201pt,1224.0977pt)[below right,color=black]{\textbf{37}};
		\node at (67.8047pt,1224.0977pt)[below right,color=black]{aaa};
		\draw[pstyle3] (52.8201pt,1270.6797pt) -- (42.8201pt,1274.6797pt) -- (52.8201pt,1278.6797pt) -- (48.8201pt,1274.6797pt) -- cycle;
		\draw[pstyle5] (46.8201pt,1274.6797pt) -- (513.4055pt,1274.6797pt);
		\node at (58.8201pt,1255.3887pt)[below right,color=black]{\textbf{38}};
		\node at (77.8047pt,1255.3887pt)[below right,color=black]{:aaa};
		\draw[pstyle3] (554.8055pt,1301.9707pt) -- (564.8055pt,1305.9707pt) -- (554.8055pt,1309.9707pt) -- (558.8055pt,1305.9707pt) -- cycle;
		\draw[pstyle4] (41.8201pt,1305.9707pt) -- (560.8055pt,1305.9707pt);
		\node at (48.8201pt,1286.6797pt)[below right,color=black]{\textbf{39}};
		\node at (67.8047pt,1286.6797pt)[below right,color=black]{aaa};
		\draw[pstyle3] (52.8201pt,1333.2617pt) -- (42.8201pt,1337.2617pt) -- (52.8201pt,1341.2617pt) -- (48.8201pt,1337.2617pt) -- cycle;
		\draw[pstyle5] (46.8201pt,1337.2617pt) -- (565.8055pt,1337.2617pt);
		\node at (58.8201pt,1317.9707pt)[below right,color=black]{\textbf{40}};
		\node at (77.8047pt,1317.9707pt)[below right,color=black]{:aaa};
		\draw[pstyle3] (605.8426pt,1364.5527pt) -- (615.8426pt,1368.5527pt) -- (605.8426pt,1372.5527pt) -- (609.8426pt,1368.5527pt) -- cycle;
		\draw[pstyle4] (41.8201pt,1368.5527pt) -- (611.8426pt,1368.5527pt);
		\node at (48.8201pt,1349.2617pt)[below right,color=black]{\textbf{41}};
		\node at (67.8047pt,1349.2617pt)[below right,color=black]{aaa};
		\draw[pstyle3] (52.8201pt,1395.8438pt) -- (42.8201pt,1399.8438pt) -- (52.8201pt,1403.8438pt) -- (48.8201pt,1399.8438pt) -- cycle;
		\draw[pstyle5] (46.8201pt,1399.8438pt) -- (616.8426pt,1399.8438pt);
		\node at (58.8201pt,1380.5527pt)[below right,color=black]{\textbf{42}};
		\node at (77.8047pt,1380.5527pt)[below right,color=black]{:aaa};
		\draw[pstyle3] (502.4055pt,1427.1348pt) -- (512.4055pt,1431.1348pt) -- (502.4055pt,1435.1348pt) -- (506.4055pt,1431.1348pt) -- cycle;
		\draw[pstyle4] (41.8201pt,1431.1348pt) -- (508.4055pt,1431.1348pt);
		\node at (48.8201pt,1411.8438pt)[below right,color=black]{\textbf{43}};
		\node at (67.8047pt,1411.8438pt)[below right,color=black]{aaa};
		\draw[pstyle3] (52.8201pt,1458.4258pt) -- (42.8201pt,1462.4258pt) -- (52.8201pt,1466.4258pt) -- (48.8201pt,1462.4258pt) -- cycle;
		\draw[pstyle5] (46.8201pt,1462.4258pt) -- (513.4055pt,1462.4258pt);
		\node at (58.8201pt,1443.1348pt)[below right,color=black]{\textbf{44}};
		\node at (77.8047pt,1443.1348pt)[below right,color=black]{:aaa};
		\draw[pstyle3] (660.9368pt,1489.7168pt) -- (670.9368pt,1493.7168pt) -- (660.9368pt,1497.7168pt) -- (664.9368pt,1493.7168pt) -- cycle;
		\draw[pstyle4] (41.8201pt,1493.7168pt) -- (666.9368pt,1493.7168pt);
		\node at (48.8201pt,1474.4258pt)[below right,color=black]{\textbf{45}};
		\node at (67.8047pt,1474.4258pt)[below right,color=black]{aaa};
		\draw[pstyle3] (52.8201pt,1521.0078pt) -- (42.8201pt,1525.0078pt) -- (52.8201pt,1529.0078pt) -- (48.8201pt,1525.0078pt) -- cycle;
		\draw[pstyle5] (46.8201pt,1525.0078pt) -- (671.9368pt,1525.0078pt);
		\node at (58.8201pt,1505.7168pt)[below right,color=black]{\textbf{46}};
		\node at (77.8047pt,1505.7168pt)[below right,color=black]{:aaa};
		\draw[pstyle3] (96.9206pt,1552.2988pt) -- (106.9206pt,1556.2988pt) -- (96.9206pt,1560.2988pt) -- (100.9206pt,1556.2988pt) -- cycle;
		\draw[pstyle4] (41.8201pt,1556.2988pt) -- (102.9206pt,1556.2988pt);
		\node at (48.8201pt,1537.0078pt)[below right,color=black]{\textbf{47}};
		\node at (67.8047pt,1537.0078pt)[below right,color=black]{aaa};
		\draw[pstyle3] (52.8201pt,1583.5898pt) -- (42.8201pt,1587.5898pt) -- (52.8201pt,1591.5898pt) -- (48.8201pt,1587.5898pt) -- cycle;
		\draw[pstyle5] (46.8201pt,1587.5898pt) -- (107.9206pt,1587.5898pt);
		\node at (58.8201pt,1568.2988pt)[below right,color=black]{\textbf{48}};
		\node at (77.8047pt,1568.2988pt)[below right,color=black]{:aaa};
		\draw[pstyle3] (96.9206pt,1614.8809pt) -- (106.9206pt,1618.8809pt) -- (96.9206pt,1622.8809pt) -- (100.9206pt,1618.8809pt) -- cycle;
		\draw[pstyle4] (41.8201pt,1618.8809pt) -- (102.9206pt,1618.8809pt);
		\node at (48.8201pt,1599.5898pt)[below right,color=black]{\textbf{49}};
		\node at (67.8047pt,1599.5898pt)[below right,color=black]{aaa};
		\draw[pstyle3] (52.8201pt,1646.1719pt) -- (42.8201pt,1650.1719pt) -- (52.8201pt,1654.1719pt) -- (48.8201pt,1650.1719pt) -- cycle;
		\draw[pstyle5] (46.8201pt,1650.1719pt) -- (107.9206pt,1650.1719pt);
		\node at (58.8201pt,1630.8809pt)[below right,color=black]{\textbf{50}};
		\node at (77.8047pt,1630.8809pt)[below right,color=black]{:aaa};
		\draw[pstyle3] (96.9206pt,1677.4629pt) -- (106.9206pt,1681.4629pt) -- (96.9206pt,1685.4629pt) -- (100.9206pt,1681.4629pt) -- cycle;
		\draw[pstyle4] (41.8201pt,1681.4629pt) -- (102.9206pt,1681.4629pt);
		\node at (48.8201pt,1662.1719pt)[below right,color=black]{\textbf{51}};
		\node at (67.8047pt,1662.1719pt)[below right,color=black]{aaa};
		\draw[pstyle3] (52.8201pt,1708.7539pt) -- (42.8201pt,1712.7539pt) -- (52.8201pt,1716.7539pt) -- (48.8201pt,1712.7539pt) -- cycle;
		\draw[pstyle5] (46.8201pt,1712.7539pt) -- (107.9206pt,1712.7539pt);
		\node at (58.8201pt,1693.4629pt)[below right,color=black]{\textbf{52}};
		\node at (77.8047pt,1693.4629pt)[below right,color=black]{:aaa};
		\draw[pstyle3] (96.9206pt,1740.0449pt) -- (106.9206pt,1744.0449pt) -- (96.9206pt,1748.0449pt) -- (100.9206pt,1744.0449pt) -- cycle;
		\draw[pstyle4] (41.8201pt,1744.0449pt) -- (102.9206pt,1744.0449pt);
		\node at (48.8201pt,1724.7539pt)[below right,color=black]{\textbf{53}};
		\node at (67.8047pt,1724.7539pt)[below right,color=black]{aaa};
		\draw[pstyle3] (52.8201pt,1771.3359pt) -- (42.8201pt,1775.3359pt) -- (52.8201pt,1779.3359pt) -- (48.8201pt,1775.3359pt) -- cycle;
		\draw[pstyle5] (46.8201pt,1775.3359pt) -- (107.9206pt,1775.3359pt);
		\node at (58.8201pt,1756.0449pt)[below right,color=black]{\textbf{54}};
		\node at (77.8047pt,1756.0449pt)[below right,color=black]{:aaa};
		\draw[pstyle3] (96.9206pt,1802.627pt) -- (106.9206pt,1806.627pt) -- (96.9206pt,1810.627pt) -- (100.9206pt,1806.627pt) -- cycle;
		\draw[pstyle4] (41.8201pt,1806.627pt) -- (102.9206pt,1806.627pt);
		\node at (48.8201pt,1787.3359pt)[below right,color=black]{\textbf{55}};
		\node at (67.8047pt,1787.3359pt)[below right,color=black]{aaa};
		\draw[pstyle3] (52.8201pt,1833.918pt) -- (42.8201pt,1837.918pt) -- (52.8201pt,1841.918pt) -- (48.8201pt,1837.918pt) -- cycle;
		\draw[pstyle5] (46.8201pt,1837.918pt) -- (107.9206pt,1837.918pt);
		\node at (58.8201pt,1818.627pt)[below right,color=black]{\textbf{56}};
		\node at (77.8047pt,1818.627pt)[below right,color=black]{:aaa};
		\draw[pstyle3] (96.9206pt,1865.209pt) -- (106.9206pt,1869.209pt) -- (96.9206pt,1873.209pt) -- (100.9206pt,1869.209pt) -- cycle;
		\draw[pstyle4] (41.8201pt,1869.209pt) -- (102.9206pt,1869.209pt);
		\node at (48.8201pt,1849.918pt)[below right,color=black]{\textbf{57}};
		\node at (67.8047pt,1849.918pt)[below right,color=black]{aaa};
		\draw[pstyle3] (52.8201pt,1896.5pt) -- (42.8201pt,1900.5pt) -- (52.8201pt,1904.5pt) -- (48.8201pt,1900.5pt) -- cycle;
		\draw[pstyle5] (46.8201pt,1900.5pt) -- (107.9206pt,1900.5pt);
		\node at (58.8201pt,1881.209pt)[below right,color=black]{\textbf{58}};
		\node at (77.8047pt,1881.209pt)[below right,color=black]{:aaa};
		\draw[pstyle3] (96.9206pt,1927.791pt) -- (106.9206pt,1931.791pt) -- (96.9206pt,1935.791pt) -- (100.9206pt,1931.791pt) -- cycle;
		\draw[pstyle4] (41.8201pt,1931.791pt) -- (102.9206pt,1931.791pt);
		\node at (48.8201pt,1912.5pt)[below right,color=black]{\textbf{59}};
		\node at (67.8047pt,1912.5pt)[below right,color=black]{aaa};
		\draw[pstyle3] (52.8201pt,1959.082pt) -- (42.8201pt,1963.082pt) -- (52.8201pt,1967.082pt) -- (48.8201pt,1963.082pt) -- cycle;
		\draw[pstyle5] (46.8201pt,1963.082pt) -- (107.9206pt,1963.082pt);
		\node at (58.8201pt,1943.791pt)[below right,color=black]{\textbf{60}};
		\node at (77.8047pt,1943.791pt)[below right,color=black]{:aaa};
		\draw[pstyle3] (96.9206pt,1990.373pt) -- (106.9206pt,1994.373pt) -- (96.9206pt,1998.373pt) -- (100.9206pt,1994.373pt) -- cycle;
		\draw[pstyle4] (41.8201pt,1994.373pt) -- (102.9206pt,1994.373pt);
		\node at (48.8201pt,1975.082pt)[below right,color=black]{\textbf{61}};
		\node at (67.8047pt,1975.082pt)[below right,color=black]{aaa};
		\draw[pstyle3] (52.8201pt,2021.6641pt) -- (42.8201pt,2025.6641pt) -- (52.8201pt,2029.6641pt) -- (48.8201pt,2025.6641pt) -- cycle;
		\draw[pstyle5] (46.8201pt,2025.6641pt) -- (107.9206pt,2025.6641pt);
		\node at (58.8201pt,2006.373pt)[below right,color=black]{\textbf{62}};
		\node at (77.8047pt,2006.373pt)[below right,color=black]{:aaa};
		\draw[pstyle3] (96.9206pt,2052.9551pt) -- (106.9206pt,2056.9551pt) -- (96.9206pt,2060.9551pt) -- (100.9206pt,2056.9551pt) -- cycle;
		\draw[pstyle4] (41.8201pt,2056.9551pt) -- (102.9206pt,2056.9551pt);
		\node at (48.8201pt,2037.6641pt)[below right,color=black]{\textbf{63}};
		\node at (67.8047pt,2037.6641pt)[below right,color=black]{aaa};
		\draw[pstyle3] (52.8201pt,2084.2461pt) -- (42.8201pt,2088.2461pt) -- (52.8201pt,2092.2461pt) -- (48.8201pt,2088.2461pt) -- cycle;
		\draw[pstyle5] (46.8201pt,2088.2461pt) -- (107.9206pt,2088.2461pt);
		\node at (58.8201pt,2068.9551pt)[below right,color=black]{\textbf{64}};
		\node at (77.8047pt,2068.9551pt)[below right,color=black]{:aaa};
		\draw[pstyle3] (96.9206pt,2115.5371pt) -- (106.9206pt,2119.5371pt) -- (96.9206pt,2123.5371pt) -- (100.9206pt,2119.5371pt) -- cycle;
		\draw[pstyle4] (41.8201pt,2119.5371pt) -- (102.9206pt,2119.5371pt);
		\node at (48.8201pt,2100.2461pt)[below right,color=black]{\textbf{65}};
		\node at (67.8047pt,2100.2461pt)[below right,color=black]{aaa};
		\draw[pstyle3] (52.8201pt,2146.8281pt) -- (42.8201pt,2150.8281pt) -- (52.8201pt,2154.8281pt) -- (48.8201pt,2150.8281pt) -- cycle;
		\draw[pstyle5] (46.8201pt,2150.8281pt) -- (107.9206pt,2150.8281pt);
		\node at (58.8201pt,2131.5371pt)[below right,color=black]{\textbf{66}};
		\node at (77.8047pt,2131.5371pt)[below right,color=black]{:aaa};
		\draw[pstyle3] (96.9206pt,2178.1191pt) -- (106.9206pt,2182.1191pt) -- (96.9206pt,2186.1191pt) -- (100.9206pt,2182.1191pt) -- cycle;
		\draw[pstyle4] (41.8201pt,2182.1191pt) -- (102.9206pt,2182.1191pt);
		\node at (48.8201pt,2162.8281pt)[below right,color=black]{\textbf{67}};
		\node at (67.8047pt,2162.8281pt)[below right,color=black]{aaa};
		\draw[pstyle3] (52.8201pt,2209.4102pt) -- (42.8201pt,2213.4102pt) -- (52.8201pt,2217.4102pt) -- (48.8201pt,2213.4102pt) -- cycle;
		\draw[pstyle5] (46.8201pt,2213.4102pt) -- (107.9206pt,2213.4102pt);
		\node at (58.8201pt,2194.1191pt)[below right,color=black]{\textbf{68}};
		\node at (77.8047pt,2194.1191pt)[below right,color=black]{:aaa};
		\draw[pstyle3] (96.9206pt,2240.7012pt) -- (106.9206pt,2244.7012pt) -- (96.9206pt,2248.7012pt) -- (100.9206pt,2244.7012pt) -- cycle;
		\draw[pstyle4] (41.8201pt,2244.7012pt) -- (102.9206pt,2244.7012pt);
		\node at (48.8201pt,2225.4102pt)[below right,color=black]{\textbf{69}};
		\node at (67.8047pt,2225.4102pt)[below right,color=black]{aaa};
		\draw[pstyle3] (52.8201pt,2271.9922pt) -- (42.8201pt,2275.9922pt) -- (52.8201pt,2279.9922pt) -- (48.8201pt,2275.9922pt) -- cycle;
		\draw[pstyle5] (46.8201pt,2275.9922pt) -- (107.9206pt,2275.9922pt);
		\node at (58.8201pt,2256.7012pt)[below right,color=black]{\textbf{70}};
		\node at (77.8047pt,2256.7012pt)[below right,color=black]{:aaa};
		\draw[pstyle3] (96.9206pt,2303.2832pt) -- (106.9206pt,2307.2832pt) -- (96.9206pt,2311.2832pt) -- (100.9206pt,2307.2832pt) -- cycle;
		\draw[pstyle4] (41.8201pt,2307.2832pt) -- (102.9206pt,2307.2832pt);
		\node at (48.8201pt,2287.9922pt)[below right,color=black]{\textbf{71}};
		\node at (67.8047pt,2287.9922pt)[below right,color=black]{aaa};
		\draw[pstyle3] (52.8201pt,2334.5742pt) -- (42.8201pt,2338.5742pt) -- (52.8201pt,2342.5742pt) -- (48.8201pt,2338.5742pt) -- cycle;
		\draw[pstyle5] (46.8201pt,2338.5742pt) -- (107.9206pt,2338.5742pt);
		\node at (58.8201pt,2319.2832pt)[below right,color=black]{\textbf{72}};
		\node at (77.8047pt,2319.2832pt)[below right,color=black]{:aaa};
		\draw[pstyle3] (96.9206pt,2365.8652pt) -- (106.9206pt,2369.8652pt) -- (96.9206pt,2373.8652pt) -- (100.9206pt,2369.8652pt) -- cycle;
		\draw[pstyle4] (41.8201pt,2369.8652pt) -- (102.9206pt,2369.8652pt);
		\node at (48.8201pt,2350.5742pt)[below right,color=black]{\textbf{73}};
		\node at (67.8047pt,2350.5742pt)[below right,color=black]{aaa};
		\draw[pstyle3] (52.8201pt,2397.1563pt) -- (42.8201pt,2401.1563pt) -- (52.8201pt,2405.1563pt) -- (48.8201pt,2401.1563pt) -- cycle;
		\draw[pstyle5] (46.8201pt,2401.1563pt) -- (107.9206pt,2401.1563pt);
		\node at (58.8201pt,2381.8652pt)[below right,color=black]{\textbf{74}};
		\node at (77.8047pt,2381.8652pt)[below right,color=black]{:aaa};
		\draw[pstyle3] (96.9206pt,2428.4473pt) -- (106.9206pt,2432.4473pt) -- (96.9206pt,2436.4473pt) -- (100.9206pt,2432.4473pt) -- cycle;
		\draw[pstyle4] (41.8201pt,2432.4473pt) -- (102.9206pt,2432.4473pt);
		\node at (48.8201pt,2413.1563pt)[below right,color=black]{\textbf{75}};
		\node at (67.8047pt,2413.1563pt)[below right,color=black]{aaa};
		\draw[pstyle3] (52.8201pt,2459.7383pt) -- (42.8201pt,2463.7383pt) -- (52.8201pt,2467.7383pt) -- (48.8201pt,2463.7383pt) -- cycle;
		\draw[pstyle5] (46.8201pt,2463.7383pt) -- (107.9206pt,2463.7383pt);
		\node at (58.8201pt,2444.4473pt)[below right,color=black]{\textbf{76}};
		\node at (77.8047pt,2444.4473pt)[below right,color=black]{:aaa};
		\draw[pstyle3] (96.9206pt,2491.0293pt) -- (106.9206pt,2495.0293pt) -- (96.9206pt,2499.0293pt) -- (100.9206pt,2495.0293pt) -- cycle;
		\draw[pstyle4] (41.8201pt,2495.0293pt) -- (102.9206pt,2495.0293pt);
		\node at (48.8201pt,2475.7383pt)[below right,color=black]{\textbf{77}};
		\node at (67.8047pt,2475.7383pt)[below right,color=black]{aaa};
		\draw[pstyle3] (52.8201pt,2522.3203pt) -- (42.8201pt,2526.3203pt) -- (52.8201pt,2530.3203pt) -- (48.8201pt,2526.3203pt) -- cycle;
		\draw[pstyle5] (46.8201pt,2526.3203pt) -- (107.9206pt,2526.3203pt);
		\node at (58.8201pt,2507.0293pt)[below right,color=black]{\textbf{78}};
		\node at (77.8047pt,2507.0293pt)[below right,color=black]{:aaa};
		\draw[pstyle3] (96.9206pt,2553.6113pt) -- (106.9206pt,2557.6113pt) -- (96.9206pt,2561.6113pt) -- (100.9206pt,2557.6113pt) -- cycle;
		\draw[pstyle4] (41.8201pt,2557.6113pt) -- (102.9206pt,2557.6113pt);
		\node at (48.8201pt,2538.3203pt)[below right,color=black]{\textbf{79}};
		\node at (67.8047pt,2538.3203pt)[below right,color=black]{aaa};
		\draw[pstyle3] (52.8201pt,2584.9023pt) -- (42.8201pt,2588.9023pt) -- (52.8201pt,2592.9023pt) -- (48.8201pt,2588.9023pt) -- cycle;
		\draw[pstyle5] (46.8201pt,2588.9023pt) -- (107.9206pt,2588.9023pt);
		\node at (58.8201pt,2569.6113pt)[below right,color=black]{\textbf{80}};
		\node at (77.8047pt,2569.6113pt)[below right,color=black]{:aaa};
	\end{tikzpicture}
}

\end{lrbox}%
\noindent{
\setlength{\remainingheight}{\textheight}
\addtolength{\remainingheight}{-\ht\strutbox}
\ifdim\ht\mysavebox>\remainingheight
\setlength{\myrest}{\ht\mysavebox}%
\clipbox{0 {\myrest-\remainingheight} 0 {\ht\mysavebox-\myrest}}{\usebox{\mysavebox}}%
\addtolength{\myrest}{-\remainingheight}%
\loop\ifdim\myrest>\textheight
\newpage\par\noindent
\clipbox{0 {\myrest-\textheight} 0 {\ht\mysavebox-\myrest}}{\usebox{\mysavebox}}%
\addtolength{\myrest}{-\textheight}%
\repeat
\newpage\par\noindent
\clipbox{0 0 0 {\ht\mysavebox-\myrest}}{\usebox{\mysavebox}}%
\else
\usebox{\mysavebox}%
\fi
}
\chapter{实验}


\backmatter % 参考文献之前
\chapter{主要结论}
\bibliographystyle{gbt7714-numerical}
\bibliography{data/refs}% 如果参考文献中有&需要展示,应该使用\& 参考文献标题如需实词全首字母大写,自行修改bib
%\Appendix

\Work

\noindent
\textbf{发表学术论文}
\begin{enumerate}
	\renewcommand{\labelenumi}{[\theenumi]}
	\item Chen J, Dou S, Zhang W. Binaural rendering based on linear differential microphone array and ambisonic reproduction[J]. Journal of Fudan University(Natural Science),2019, 58(3):370–377.
\end{enumerate}

~\\

\noindent
\textbf{发表专利}
\begin{enumerate}
	\renewcommand{\labelenumi}{[\theenumi]}
	\item 专利名称:一种混合型主动降噪方法与系统(已受理,第一发明人)。
	\item 专利名称:一种主动降噪耳机的滤波器设计方法(已受理)。
\end{enumerate}%个人成果

\Thanks

时光飞逝,转眼间已经要说再见,在西工大度过的七年时光历历在目,有着太多的不舍。学习和成长的过程,离不开老师的悉心教导和同学的关心帮助,在此我要向大家表示我最真诚的感谢。

首先要特别感谢我的两位指导老师:张雯教授和陈景东教授。从论文选题到研究工作的开展以及论文的撰写,都是在两位老师的指导下进行的。两位老师认真负责、精益求精的工作态度、渊博的知识储备、平易近人的处事方法深深影响了我,永远是我科研和生活的人生榜样。从本科毕业设计到现在,这三年多的时间里,我从刚开始的科研小白到现在有一定的理解和成果,都离不开两位老师的耐心指导,一次次的讨论让我对自己的研究方向和工作内容有了清晰的认知,使我的学习能力与动手能力得到了提升。老师们不仅在科研上给予帮助,还非常关心我们的生活情况和就业进展。

感谢教研室的师兄师姐和师弟师妹们,他们在整个研究生生涯中给予了我很多帮助,一起度过了很多快乐的时光。在探讨交流过程中,我吸收了很多宝贵的经验,为科研工作奠定基础。其中特别感谢郗经纬同学,在遇到问题时共同讨论、互帮互助,还有同级的陈奕彤和李梦真,在整个论文的撰写过程中相互鼓励,正是这种合作精神,给我们的研究生生涯画上了一个完美的句号。感谢张俊卿师兄在我遇到不懂的问题时给予分析和建议,对论文提出修改意见,同时感谢帮我审阅论文的师妹尤新雅、胡轩琦、师弟王佳乐,感谢你们提出的宝贵意见。感谢我硕士期间的宝藏舍友们,每天回到宿舍都会有一个开心轻松的氛围,互相关心,相信我们七年的友谊在之后的时间里可以得到更好的发展。

在此,还要感谢我的家人和男朋友对我无微不至的关心与理解。你们是我的坚强后盾,对我的任何决定都给予支持,让我可以毫无顾虑地勇往直前。在我遇到困难和压力时,你们的陪伴和鼓励使我能够保持乐观的心态,做出更好的成绩。

同时,我还要感谢我的班主任李道江老师和就业指导中心这个大家庭,在我的生涯规划和求职过程中给予指导和关心。最后还要向本领域进行相关研究的学者专家,以及各位审阅评议论文的老师表示由衷的感谢!%致谢
\statement
\end{document}